\documentclass[12pt, titlepage]{article}

\usepackage{booktabs}
\usepackage{tabularx}
\usepackage{hyperref}
\hypersetup{
    colorlinks,
    citecolor=black,
    filecolor=black,
    linkcolor=red,
    urlcolor=blue
}
\usepackage[round]{natbib}

%% Comments

\usepackage{color}

\newif\ifcomments\commentstrue %displays comments
%\newif\ifcomments\commentsfalse %so that comments do not display

\ifcomments
\newcommand{\authornote}[3]{\textcolor{#1}{[#3 ---#2]}}
\newcommand{\todo}[1]{\textcolor{red}{[TODO: #1]}}
\else
\newcommand{\authornote}[3]{}
\newcommand{\todo}[1]{}
\fi

\newcommand{\wss}[1]{\authornote{blue}{SS}{#1}} 
\newcommand{\plt}[1]{\authornote{magenta}{TPLT}{#1}} %For explanation of the template
\newcommand{\an}[1]{\authornote{cyan}{Author}{#1}}

%% Common Parts

\newcommand{\progname}{2D-RAPP} % PUT YOUR PROGRAM NAME HERE
\newcommand{\authname}{Ziyang(Ryan) Fang} % AUTHOR NAMES                  

\usepackage{hyperref}
    \hypersetup{colorlinks=true, linkcolor=blue, citecolor=blue, filecolor=blue,
                urlcolor=blue, unicode=false}
    \urlstyle{same}
                                


\begin{document}

\title{Verification and Validation Report: 2D-RAPP} 
\author{Ziyang Fang}
\date{April 18, 2025}
	
\maketitle

\pagenumbering{roman}

\section{Revision History}

\begin{tabularx}{\textwidth}{p{3cm}p{2cm}X}
\toprule {\bf Date} & {\bf Version} & {\bf Notes}\\
\midrule
April 16, 2025 & 1.0 & Initial version of the V\&V Report \\
\bottomrule
\end{tabularx}

~\newpage

\section{Symbols, Abbreviations and Acronyms}

\renewcommand{\arraystretch}{1.2}
\begin{tabular}{l l} 
    \toprule		
    \textbf{symbol} & \textbf{description}\\
    \midrule 
    A & Assumption\\
    DD & Data Definition\\
    GD & General Definition\\
    GS & Goal Statement\\
    IM & Instance Model\\
    LC & Likely Change\\
    PS & Physical System Description\\
    R & Requirement\\
    SRS & Software Requirements Specification\\
    TM & Theoretical Model\\
    IK & Inverse Kinematics \\
    FK & Forward Kinematics \\
    A* & A-star Pathfinding Algorithm \\
    DOF & Degrees of Freedom \\
    EE & End-Effector \\
    % \progname{} & 2DPL\\
    2D-RAPP & 2D Robot Arm Path Planning\\
    \bottomrule
  \end{tabular}\\

\newpage

\tableofcontents
\newpage

\pagenumbering{arabic}

\section{Functional Requirements Evaluation}

\subsection*{Collision-Free Path Generation}

\begin{tabularx}{\textwidth}{l X l}
\toprule
\textbf{Test ID} & \textbf{Description} & \textbf{Result} \\
\midrule
T1 & Basic collision-free path planning & Pass \\
T2 & Path planning with multiple obstacles & Pass \\
\bottomrule
\end{tabularx}

\textbf{Comments:} The path planner module reliably generates valid, collision-free trajectories. The A* algorithm performs well on a toroidal joint-space grid.

\subsection*{Inverse Kinematics Solver Validation}

\begin{tabularx}{\textwidth}{l X l}
\toprule
\textbf{Test ID} & \textbf{Description} & \textbf{Result} \\
\midrule
T3 & Feasibility of IK solution & Pass \\
T4 & IK for complex configurations & Pass \\
\bottomrule
\end{tabularx}

\textbf{Comments:} The IK solver computes valid and optimal solutions, even in redundant configurations.

\section{Nonfunctional Requirements Evaluation}

\subsection{Performance}

\begin{tabularx}{\textwidth}{l X l}
\toprule
\textbf{Test ID} & \textbf{Description} & \textbf{Result} \\
\midrule
N1 & Planning under high obstacle density & Pass \\
N2 & Scalability with increased DOF & Pass \\
\bottomrule
\end{tabularx}

\textbf{Comments:} The system maintains real-time performance and memory usage within acceptable limits.

\section{Comparison to Existing Implementation}

Not applicable.

\section{Unit Testing}

\begin{tabularx}{\textwidth}{l l l l}
\toprule
\textbf{Test ID} & \textbf{Module} & \textbf{Coverage} & \textbf{Result} \\
\midrule
U1 & Collision Detection & 100\% & Pass \\
U2 & IK Solver & 100\% & Pass \\
U3 & Path Planner & 100\% & Pass \\
\bottomrule
\end{tabularx}

\textbf{Comments:} Unit tests cover all edge cases, and functional outputs match expectations under various scenarios.

\section{Changes Due to Testing}

Based on using experience and personal feedback:

\begin{itemize}
  \item Refined obstacle representation for better precision.
  \item Clarified collision detection: defined tangent cases as non-colliding.
  \item Improved GUI visualization of configurations and trajectories.
\end{itemize}

\section{Automated Testing}

The following tools were employed for continuous integration and quality assurance:

\begin{itemize}
  \item \textbf{pytest} for automated unit testing.
  \item \textbf{coverage.py} to ensure high code coverage.
  \item \textbf{flake8} for code style and static analysis.
  \item \textbf{GitHub Actions} for CI on every commit.
\end{itemize}

\section{Trace to Requirements}

\begin{tabularx}{\textwidth}{l l l}
\toprule
\textbf{Requirement} & \textbf{Test Case(s)} & \textbf{Status} \\
\midrule
FR1: Obstacle avoidance & T1 & Pass \\
FR2: Multiple obstacles & T2 & Pass \\
FR3: IK feasibility & T3, T4 & Pass \\
NFR1: Performance & N1, N2 & Pass \\
\bottomrule
\end{tabularx}

\section{Trace to Modules}

\begin{tabularx}{\textwidth}{l l l}
\toprule
\textbf{Module} & \textbf{Test IDs} & \textbf{Status} \\
\midrule
Collision Detection & U1 & Pass \\
IK Solver & U2 & Pass \\
Path Planner & U3 & Pass \\
\bottomrule
\end{tabularx}

\section{Code Coverage Metrics}

\begin{tabularx}{\textwidth}{l l l l}
\toprule
\textbf{Module} & \textbf{Statements} & \textbf{Missed} & \textbf{Coverage} \\
\midrule
astar\_planner.py & 63 & 0 & 100\% \\
collision.py & 40 & 0 & 100\% \\
joint\_limits.py & 25 & 0 & 100\% \\
nlink\_arm.py & 75 & 0 & 100\% \\
\bottomrule
\end{tabularx}

\textbf{Total Coverage:} 100\%

\textbf{Comments:} All core modules are fully tested and verified.

\bibliographystyle{plainnat}
\bibliography{../../refs/References}

\newpage{}
\section*{Appendix --- Reflection}

The system design and testing activities were guided by the requirements defined in the SRS~\citep{SRS2025}, 
while the modular architecture was described in the MG~\citep{MG2025} and detailed in the MIS~\citep{MIS2025}. 
The current report follows the methodology outlined in the V\&V plan~\citep{VnV2025}.


\begin{enumerate}
  \item The testing framework and modular design made this deliverable smooth. The tests matched well with the planned architecture.
  \item Some edge cases in collision detection were challenging (e.g., tangent contacts). We resolved this by refining geometric definitions and test logic.
  \item Peer and supervisor feedback helped shape key sections like test case design and UI presentation; the rest followed internal planning.
  \item The actual V\&V activities closely followed the plan. Minor modifications were introduced during execution (e.g., visual tweaks, detection precision), which emerged from real testing scenarios. In future projects, allocating time for such edge refinement would be beneficial.
\end{enumerate}

\end{document}
