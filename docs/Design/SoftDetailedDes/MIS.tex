\documentclass[12pt, titlepage]{article}

\usepackage{amsmath, mathtools}

\usepackage[round]{natbib}
\usepackage{amsfonts}
\usepackage{amssymb}
\usepackage{graphicx}
\usepackage{colortbl}
\usepackage{xr}
\usepackage{hyperref}
\usepackage{longtable}
\usepackage{xfrac}
\usepackage{tabularx}
\usepackage{float}
\usepackage{siunitx}
\usepackage{booktabs}
\usepackage{multirow}
\usepackage[section]{placeins}
\usepackage{caption}
\usepackage{fullpage}

\hypersetup{
bookmarks=true,     % show bookmarks bar?
colorlinks=true,       % false: boxed links; true: colored links
linkcolor=red,          % color of internal links (change box color with linkbordercolor)
citecolor=blue,      % color of links to bibliography
filecolor=magenta,  % color of file links
urlcolor=cyan          % color of external links
}

\usepackage{array}

\externaldocument{../../SRS/SRS}

%% Comments

\usepackage{color}

\newif\ifcomments\commentstrue %displays comments
%\newif\ifcomments\commentsfalse %so that comments do not display

\ifcomments
\newcommand{\authornote}[3]{\textcolor{#1}{[#3 ---#2]}}
\newcommand{\todo}[1]{\textcolor{red}{[TODO: #1]}}
\else
\newcommand{\authornote}[3]{}
\newcommand{\todo}[1]{}
\fi

\newcommand{\wss}[1]{\authornote{blue}{SS}{#1}} 
\newcommand{\plt}[1]{\authornote{magenta}{TPLT}{#1}} %For explanation of the template
\newcommand{\an}[1]{\authornote{cyan}{Author}{#1}}

%% Common Parts

\newcommand{\progname}{2D-RAPP} % PUT YOUR PROGRAM NAME HERE
\newcommand{\authname}{Ziyang(Ryan) Fang} % AUTHOR NAMES                  

\usepackage{hyperref}
    \hypersetup{colorlinks=true, linkcolor=blue, citecolor=blue, filecolor=blue,
                urlcolor=blue, unicode=false}
    \urlstyle{same}
                                


\begin{document}

\title{Module Interface Specification for \progname{}}

\author{\authname}

\date{\today}

\maketitle

\pagenumbering{roman}

\section{Revision History}

\begin{tabularx}{\textwidth}{p{3cm}p{2cm}X}
\toprule {\bf Date} & {\bf Version} & {\bf Notes}\\
\midrule
2025 March 19 & 1.0 & Notes\\
\bottomrule
\end{tabularx}

~\newpage

\section{Symbols, Abbreviations and Acronyms}

See SRS Documentation at \url{https://github.com/FangZiyang/CAS741-Ryan/blob/main/docs/SRS/SRS.pdf}.


\newpage

\tableofcontents

\newpage

\pagenumbering{arabic}

\section{Introduction}


The following document details the Module Interface Specifications for
\textbf{2D Robot Arm Path Planning} 

Complementary documents include the System Requirement Specifications
and Module Guide.  The full documentation and implementation can be
found at \url{https://github.com/FangZiyang/CAS741-Ryan}. 

% This document is the Module Interface Specification (MIS) for the system. It is developed in accordance with the Software Requirements Specification (SRS) for the project. The MIS defines the modular structure of the software and the interface of each module that together fulfill the requirements for planning a collision-free path for a two-dimensional robotic arm. The system's goal, as outlined in the SRS, is to compute a safe and efficient path for the robotic arm to move from a given starting position to a desired goal position without colliding with any obstacles in its environment.

% The design described herein is implemented in \textbf{Python}, aligning with the SRS implementation constraints. Each module's responsibilities and interactions are clearly delineated to ensure the system meets the functional goals (path computation, collision avoidance, etc.) and outputs specified in the SRS. A pathfinding algorithm (such as \textit{A*}) is utilized within the Path Planning module to find an optimal route; however, the detailed mechanics of this algorithm are abstracted away in this document, focusing instead on the interface and role of the module within the system.


\section{Notation}
The structure of the MIS for modules comes from \citet{HoffmanAndStrooper1995},
with the addition that template modules have been adapted from
\cite{GhezziEtAl2003}.  The mathematical notation comes from Chapter 3 of
\citet{HoffmanAndStrooper1995}.  For instance, the symbol := is used for a
multiple assignment statement and conditional rules follow the form $(c_1
\Rightarrow r_1 | c_2 \Rightarrow r_2 | ... | c_n \Rightarrow r_n )$.

The following table summarizes the primitive data types used by \progname. 

\begin{center}
  \renewcommand{\arraystretch}{1.2}
  \noindent 
  \begin{tabular}{l l p{7.5cm}} 
  \toprule 
  \textbf{Data Type} & \textbf{Notation} & \textbf{Description}\\ 
  \midrule
  character & char & a single symbol or digit\\
  integer & $\mathbb{Z}$ & a number without a fractional component in $(-\infty, \infty)$ \\[4pt]
  natural number & $\mathbb{N}$ & a number without a fractional component in $[1, \infty)$ \\[4pt]
  real & $\mathbb{R}$ & any number in $(-\infty, \infty)$\\[4pt]

  \bottomrule
  \end{tabular} 
  \end{center}

\noindent
The specification of \progname \ uses some derived data types: sequences, strings, and
tuples. Sequences are lists filled with elements of the same data type. Strings
are sequences of characters. Tuples contain a list of values, potentially of
different types. In addition, \progname \ uses functions, which
are defined by the data types of their inputs and outputs. Local functions are
described by giving their type signature followed by their specification.

\section{Module Decomposition}

The following table is taken directly from the Module Guide document for this project.


\begin{table}[h!]
  \centering
  \begin{tabular}{p{0.3\textwidth} p{0.6\textwidth}}
  \toprule
  \textbf{Level 1} & \textbf{Level 2}\\
  \midrule
  \textbf{Hardware Hiding} & \\ 
  \midrule
  \multirow{6}{0.3\textwidth}{\textbf{Behaviour Hiding}} 
  & Input Parameters Module \\ 
  & Output Format Module \\ 
  & Output Verification Module \\ 
  & Inverse Kinematics Solver Module \\ 
  & Configuration Management Module \\ 
  & Path Planning Module \\ 
  & Collision Detection Module \\ 
  & COntroll Module\\
  \midrule
  \multirow{1}{0.3\textwidth}{\textbf{Software Decision}} 
  & Plotting Module \\ 
  \bottomrule
  \end{tabular}
  \caption{Module Hierarchy}
  \label{TblMH}
\end{table}


\newpage
% ~\newpage

\section{MIS of Path Planning Module} \label{Module:PathPlanning}

\subsection{Module}
Path Planning Module

\subsection{Uses}
This module interacts with the following components:
\begin{itemize}
\item \textbf{Input Parameters Module}: Provides start and goal configurations, obstacles, and robot parameters.
\item \textbf{Collision Detection Module}: Ensures that planned paths avoid obstacles.
\item \textbf{Output Format Module}: Handles the formatting and visualization of the planned path.
\end{itemize}

\subsection{Syntax}

\subsubsection{Exported Constants}
\begin{itemize}
  \item \texttt{MAX\_ITERATIONS} : $\mathbb{N}$ \quad (Maximum iterations for path search, default: 10,000)
  \item \texttt{HEURISTIC\_FACTOR} : $\mathbb{R}$ \quad (Scaling factor for heuristic function in A* search, default: 1.0)
\end{itemize}

\subsubsection{Exported Access Programs}
\begin{center}
  \renewcommand{\arraystretch}{1.4}
  \begin{tabular}{p{3.5cm} p{3cm} p{4cm} p{4cm}}
  \toprule
  \textbf{Name} & \textbf{In} & \textbf{Out} & \textbf{Exceptions} \\
  \midrule
  \texttt{planPath} & $\mathbb{R}^n \times \mathbb{R}^n \times Obstacles \times RobotParams$ & sequence of $\mathbb{R}^n$ & NoPathFoundException \\[4pt]
  \bottomrule
  \end{tabular}
\end{center}

\subsection{Semantics}

\subsubsection{State Variables}
None. The module does not maintain any persistent state.

\subsubsection{Environment Variables}
None. The module does not interact with external devices.

\subsubsection{Assumptions}
\begin{itemize}
  \item The robotic arm operates in a 2D plane.
  \item The environment is known beforehand and does not change dynamically.
  \item Obstacles are represented as circles with given positions and radii.
  \item The A* algorithm is used for path search in joint space.
\end{itemize}

\subsubsection{Access Routine Semantics}

\noindent \texttt{planPath}(start: \( \mathbb{R}^n \), goal: \( \mathbb{R}^n \), obstacles: \( Obstacles \), robotParams: \( RobotParams \)):

\begin{itemize}
  \item \textbf{transition}: Computes an optimal path from \(start\) to \(goal\) in joint space using A* search, ensuring that the path avoids obstacles.

  \item \textbf{output}: \( path \) is a valid and optimal sequence of robot configurations in \( \mathbb{R}^n \), satisfying the following properties:

\[
\text{isOptimalPath}(p) \equiv \forall q \in \text{validPaths}(start, goal), \quad \text{cost}(p) \leq \text{cost}(q)
\]

\[
\text{validPath}(p) \equiv \bigwedge_{i=0}^{|p|-1} \text{collisionFree}(p_i, p_{i+1}) \wedge p_0 = start \wedge p_{|p|} = goal
\]

\[
\text{cost}(p) = \sum_{i=0}^{|p|-1} \text{distance}(p_i, p_{i+1})
\]

\[
\text{heuristic}(q) = \min_{q' \in \text{neighbors}(q)} \text{distance}(q', goal)
\]

  \item \textbf{exception}: Raises \texttt{NoPathFoundException} if no valid collision-free path exists from \( start \) to \( goal \).
\end{itemize}

\subsubsection{Local Functions}


\noindent \texttt{heuristicCost}(config1: $\mathbb{R}^n$, config2: $\mathbb{R}^n$):
\begin{itemize}
  \item output: real, heuristic cost for A* search.
\end{itemize}

\noindent \texttt{distance}(config1: $\mathbb{R}^n$, config2: $\mathbb{R}^n$):
\begin{itemize}
  \item output: real, actual distance between configurations.
\end{itemize}






\section{MIS of Collision Detection Module} \label{Module:CollisionDetection}

\subsection{Module}

Collision Detection

\subsection{Uses}

This module interacts with:
\begin{itemize}
  \item \textbf{Input Parameters Module} (to obtain obstacle definitions and robot dimensions)
  \item \textbf{Path Planning Module} (to provide collision checks for planned paths)
  \item \textbf{Inverse Kinematics Solver Module} (to verify joint configurations are collision-free)
  \item \textbf{Configuration Management Module} (to access stored robot and obstacle data)
  \item \textbf{Logging and Debugging Module} (to record collision detection results for debugging)
  \end{itemize}

\subsection{Syntax}

\subsubsection{Exported Constants}

None.

\subsubsection{Exported Access Programs}

\begin{center}
  \renewcommand{\arraystretch}{1.4}
  \begin{tabular}{p{4cm} p{4cm} p{3cm} p{4cm}}
    \toprule
    \textbf{Name} & \textbf{In} & \textbf{Out} & \textbf{Exceptions} \\
    \midrule
    \texttt{checkCollision} & ADT \text{angle info} $\mathbb{R}^n \times Obstacles \times RobotParams$ & boolean & InvalidConfiguration \\[4pt]
    \texttt{checkPathCollision} & sequence of  ADT \text{angle info} $\mathbb{R}^n$ $\times$ Obstacles $\times$ RobotParams & boolean & InvalidPath \\[4pt]
    \bottomrule
  \end{tabular}
\end{center}

\subsection{Semantics}

\subsubsection{State Variables}

None. This module does not maintain any persistent state.

\subsubsection{Environment Variables}

None. This module does not interact with external devices or environments.

\subsubsection{Assumptions}

\begin{itemize}
  \item Obstacles are defined as circles with a given position and radius.
  \item Robot arm links are modeled as line segments between joint positions.
  \item The environment and obstacle positions remain static during collision detection.
\end{itemize}

\subsubsection{Access Routine Semantics}

\noindent \texttt{checkCollision}($config: \mathbb{R}^n$, $obstacles: Obstacles$, $robotParams: RobotParams$):
\begin{itemize}
\item transition: Computes whether the given configuration results in a collision by applying forward kinematics to determine the position of each joint and link. The joint positions are calculated as:

\[
p_0 = (0, 0), \quad p_i = p_{i-1} + L_i \begin{bmatrix} \cos(\theta_i) \\ \sin(\theta_i) \end{bmatrix}, \quad \forall i \in \{1, ..., n\}
\]

where:
  - \( p_i \) is the position of the \( i \)-th joint,
  - \( L_i \) is the link length,
  - \( \theta_i \) is the joint angle.

For each obstacle \( O_k = (x_k, y_k, r_k) \), the minimum distance between each robot link segment and the obstacle is computed:

\[
d = \frac{| (x_2 - x_1)(y_1 - y_k) - (y_2 - y_1)(x_1 - x_k) |}{\sqrt{(x_2 - x_1)^2 + (y_2 - y_1)^2}}
\]

where \( (x_1, y_1) \) and \( (x_2, y_2) \) are endpoints of a robot link. A collision occurs if \( d \leq r_k \).

\item output: Returns \texttt{true} if any link collides with an obstacle; otherwise, returns \texttt{false}.
\item exception: Raises \texttt{InvalidConfiguration} if the configuration violates joint constraints or is out of the robot's feasible workspace.
\end{itemize}

\noindent \texttt{checkPathCollision}($path: sequence\ of\ \mathbb{R}^n$, $obstacles: Obstacles$, $robotParams: RobotParams$):
\begin{itemize}
\item transition: Iterates over a given discrete sequence of configurations \( path = [q_1, q_2, ..., q_m] \), checking each configuration \( q_i \) using \texttt{checkCollision}. The path is validated using:

\[
\forall q_i \in path, \quad \texttt{checkCollision}(q_i, obstacles, robotParams) = false
\]

If any configuration \( q_i \) fails this check, the path is marked as invalid.

\item output: Returns \texttt{true} if any configuration in the provided path results in a collision; otherwise, returns \texttt{false}.
\item exception: Raises \texttt{InvalidPath} if the provided path contains invalid configurations or does not conform to robot motion constraints.
\end{itemize}


\subsubsection{Local Functions}

The following functions are local specification tools that clarify the module's collision-checking logic:

\noindent \texttt{forwardKinematics}($config: \mathbb{R}^n$, $robotParams: RobotParams$):
\begin{itemize}
  \item output: A sequence of joint positions representing the locations of all joints and the end-effector, used for collision checking.
\end{itemize}

\noindent \texttt{detectSegmentCollision}($pointA: \mathbb{R}^2$, $pointB: \mathbb{R}^2$, $obstacle: (\mathbb{R}^2, \mathbb{R})$):
\begin{itemize}
  \item output: \texttt{true} if the line segment defined by \texttt{pointA} and \texttt{pointB} intersects the obstacle; \texttt{false} otherwise.
\end{itemize}



\newpage

\section{MIS of Input Parameters Module} \label{Module:InputParameters}

\subsection{Module}

Input Parameters Module

\subsection{Uses}

This module interacts with:
\begin{itemize}
    \item \textbf{Configuration Management Module} (for validating and managing stored parameters)
    \item \textbf{Path Planning Module} (to provide robot configuration, obstacles, and joint limits)
    \item \textbf{Collision Detection Module} (to verify that loaded parameters do not lead to invalid configurations)
\end{itemize}

\subsection{Syntax}

\subsubsection{Exported Constants}

None.

\subsubsection{Exported Access Programs}

\begin{center}
  \renewcommand{\arraystretch}{1.4}
  \begin{tabular}{p{3cm} p{4cm} p{4cm} p{3cm}}
    \toprule
    \textbf{Name} & \textbf{In} & \textbf{Out} & \textbf{Exceptions} \\
    \midrule
    \texttt{loadParams} & ADT & - & FileNotFound, InvalidFormat, MissingParameter, InvalidValue \\[4pt]
    \texttt{getRobotParams} & - & ADT & - \\[4pt]
    \texttt{getObstacles} & - & ADT & - \\[4pt]
    \bottomrule
  \end{tabular}
\end{center}

\subsection{Semantics}

\subsubsection{State Variables}

\begin{itemize}
    \item \( RobotParams \) : Stores the dimensions and joint constraints of the robot.
    \item \( Obstacles \) : A set of obstacles represented as tuples \( (center, radius) \).
    \item \( InitialConfig \) : The starting configuration of the robot in joint space.
    \item \( GoalConfig \) : The goal configuration of the robot in joint space.
\end{itemize}

\subsubsection{Environment Variables}

\begin{itemize}
    \item \textbf{inputFile}: sequence of ADT, where each ADT corresponds to a parameter read from an external file.
\end{itemize}

\subsubsection{Assumptions}

\begin{itemize}
    \item The input file contains properly formatted parameters.
    \item The input file follows a specific order in listing parameters.
\end{itemize}

\subsubsection{Access Routine Semantics}

\noindent \texttt{loadParams}(filePath: string):
\begin{itemize}
  \item transition: Loads robot parameters, obstacles, and configurations from the specified file.
  \item exception: Raises \texttt{FileNotFound} if the file does not exist. Raises \texttt{InvalidFormat}, \texttt{MissingParameter}, or \texttt{InvalidValue} if any input is invalid.
\end{itemize}

\noindent \texttt{getRobotParams}():
\begin{itemize}
  \item output: Returns \( RobotParams \), a structure containing the robot's dimensions and joint constraints.
\end{itemize}

\subsubsection{Local Functions}

\noindent \texttt{parseParameter}:
\begin{itemize}
  \item output: Extracts numerical values from the input string and converts them to the appropriate type.
\end{itemize}


\newpage





\section{MIS of Inverse Kinematics Solver Module} \label{Module:InverseKinematics}

\subsection{Module}

Inverse Kinematics Solver

\subsection{Uses}

This module interacts with:
\begin{itemize}
    \item \textbf{Input Parameters Module} (to obtain robot kinematic parameters and constraints)
    \item \textbf{Path Planning Module} (to generate valid joint-space trajectories)
    \item \textbf{Collision Detection Module} (to ensure the computed inverse kinematics solutions do not result in collisions)
\end{itemize}

\subsection{Syntax}

\subsubsection{Exported Constants}

None.

\subsubsection{Exported Access Programs}

\begin{center}
  \renewcommand{\arraystretch}{1.4}
  \begin{tabular}{p{3.5cm} p{4cm} p{4cm} p{3cm}}
    \toprule
    \textbf{Name} & \textbf{In} & \textbf{Out} & \textbf{Exceptions} \\
    \midrule
    \texttt{solveIK} & $point \mathbb{R}^2 \times RobotParams$ & $\text{Seq}(ADT(\text{AngleInfo}))$ & NoSolutionException \\[4pt]

    \bottomrule
  \end{tabular}
\end{center}

\subsection{Semantics}

\subsubsection{State Variables}

None. This module does not maintain any persistent state.

\subsubsection{Environment Variables}

None. This module does not interact with external devices.

\subsubsection{Assumptions}

\begin{itemize}
    \item The forward kinematics function \( FK \) is well-defined and differentiable.
    \item The workspace coordinates provided are within the robot’s reachable region.
    \item The Jacobian matrix is not singular when computing inverse kinematics.
\end{itemize}

\subsubsection{Access Routine Semantics}

\noindent \texttt{solveIK}(target: $\mathbb{R}^m$, robotParams: RobotParams):
\begin{itemize}
    \item transition: Computes the joint configuration \( q \) that satisfies the inverse kinematics equation:

    \[
    FK(q) = target
    \]

    where \( FK: \mathbb{R}^n \to \mathbb{R}^m \) is the forward kinematics function of the robot. In the case where \( FK^{-1} \) is not explicitly defined, an iterative numerical method is used to approximate \( q \).

    Given an initial guess \( q_0 \), the update rule follows Newton-Raphson iteration:

    \[
    q_{k+1} = q_k + J(q_k)^{-1} (target - FK(q_k))
    \]

    where \( J(q) = \frac{\partial FK(q)}{\partial q} \) is the Jacobian matrix.

    If \( J(q) \) is singular or near-singular, a damped least-squares (DLS) method is applied:

    \[
    q_{k+1} = q_k + J(q_k)^T (J(q_k) J(q_k)^T + \lambda^2 I)^{-1} (target - FK(q_k))
    \]

    where \( \lambda \) is a small damping factor.

    \item output: A sequence of valid joint-space solutions \( q \) in \( \mathbb{R}^n \) such that:

    \[
    \| FK(q) - target \| \leq \epsilon
    \]

    where \( \epsilon \) is a predefined tolerance for numerical accuracy.

    \item exception: Raises \texttt{NoSolutionException} if no valid joint-space solution exists within the given iteration limit.
\end{itemize}



\subsubsection{Local Functions}

\noindent \texttt{forwardKinematics}(q: $\mathbb{R}^n$, robotParams: RobotParams):
\begin{itemize}
  \item output: Returns the corresponding workspace coordinates \( FK(q) \).
\end{itemize}

\noindent \texttt{jacobianMatrix}(q: $\mathbb{R}^n$, robotParams: RobotParams):
\begin{itemize}
  \item output: Computes the Jacobian matrix \( J(q) = \frac{\partial FK(q)}{\partial q} \), used in numerical inverse kinematics solutions.
\end{itemize}

\noindent \texttt{numericalIK}(target: $\mathbb{R}^m$, robotParams: RobotParams):
\begin{itemize}
  \item output: Uses an iterative method (such as Newton-Raphson) to compute an approximate inverse kinematics solution.
\end{itemize}

\newpage

\section{MIS of Output Verification Module} \label{Module:OutputVerification}

\subsection{Module}

Output Verification Module

\subsection{Uses}

This module interacts with:
\begin{itemize}
    \item \textbf{Path Planning Module} (to verify the correctness of planned paths)
    \item \textbf{Collision Detection Module} (to check if the final path avoids collisions)
    \item \textbf{Input Parameters Module} (to access the input parameters and thresholds for validation)
\end{itemize}

\subsection{Syntax}

\subsubsection{Exported Constants}

\begin{itemize}
    \item \texttt{ADMIS\_ER} = \(1 \times 10^{-6}\) (Tolerance threshold for acceptable numerical errors)
\end{itemize}

\subsubsection{Exported Access Programs}

\begin{center}
\begin{tabular}{p{4cm} p{6cm} p{3cm} p{3cm}}
    \toprule
    \textbf{Name} & \textbf{In} & \textbf{Out} & \textbf{Exceptions} \\
    \midrule
    \texttt{verifyOutput} & $path: ADT(Path), obstacles: ADT(Obstacles), robotParams: ADT(RobotParams)$ & - & PATH\_INVALID, COLLISION\_DETECTED \\[4pt]
    \bottomrule
\end{tabular}
\end{center}

\subsection{Semantics}

\subsubsection{State Variables}

None. This module does not maintain any persistent state.

\subsubsection{Environment Variables}

None. This module does not interact with external devices.

\subsubsection{Assumptions}

\begin{itemize}
    \item The computed path is a discrete sequence of configurations in \( \mathbb{R}^n \).
    \item Obstacles are represented as closed regions in \( \mathbb{R}^2 \) or \( \mathbb{R}^3 \).
    \item The verification module ensures that the planned path is both valid and collision-free.
    \item The energy conservation check ensures numerical accuracy.
\end{itemize}

\subsubsection{Access Routine Semantics}

\noindent \texttt{verifyOutput}$(path, obstacles, robotParams)$:
\begin{itemize}
    \item transition: Validates the computed path to ensure:
    \[
    \forall q_i \in path, \quad q_i \text{ satisfies joint constraints and does not intersect obstacles.}
    \]
    \item output: If the path passes all verification checks, the module does not return an explicit output.
    \item exception: Raises:
    \begin{itemize}
        \item \texttt{PATH\_INVALID} if any configuration \( q_i \) violates kinematic constraints.
        \item \texttt{COLLISION\_DETECTED} if any segment in the path intersects an obstacle:
        \[
        \exists i, \quad detectCollision(q_i, obstacles) = \text{true}.
        \]
    \end{itemize}
\end{itemize}

\subsubsection{Local Functions}

\noindent \texttt{detectCollision}$(q, obstacles)$:
\begin{itemize}
    \item output: Returns \texttt{true} if \( q \) results in a collision with any obstacle; otherwise, returns \texttt{false}.
\end{itemize}

\noindent \texttt{computePathError}$(path)$:
\begin{itemize}
    \item output: Computes the total deviation in path smoothness:
    \[
    \sum_{i=0}^{n-2} ||q_{i+1} - q_i||.
    \]
\end{itemize}

\newpage

\section{MIS of Plotting Module} \label{Module:Plotting}

\subsection{Module}

Plotting

\subsection{Uses}

This module interacts with:
\begin{itemize}

    \item \textbf{Output Format Module} (to format the the data that is correctly structured before plotting)
\end{itemize}

\subsection{Syntax}

\subsubsection{Exported Constants}

None.

\subsubsection{Exported Access Programs}

\begin{center}
\begin{tabular}{p{3.5cm} p{6cm} p{3cm} p{3cm}}
    \toprule
    \textbf{Name} & \textbf{In} & \textbf{Out} & \textbf{Exceptions} \\
    \midrule
    \texttt{plot} & $point R^{2}$, $RobotParams$, $Obstacles$  & - & - \\[4pt]

    \bottomrule
\end{tabular}
\end{center}

\subsection{Semantics}

\subsubsection{State Variables}

None. This module does not maintain any persistent state.

\subsubsection{Environment Variables}

\begin{itemize}
    \item \texttt{win}: 2D graphical window displaying the plot.
\end{itemize}

\subsubsection{Assumptions}

\begin{itemize}
    \item The plotting module receives validated input data.
    \item The visualization environment supports real-time rendering.
    \item The plotted path and metrics remain static during execution.
\end{itemize}

\subsubsection{Access Routine Semantics}

\noindent \texttt{plotPath}$(path)$:
\begin{itemize}
    \item transition: Reads the planned path and generates a 2D trajectory plot using Python's Matplotlib.
    \item output: Displays the robot's motion along the planned path, with:
    \begin{itemize}
        \item Waypoints marked as circles.
        \item The robot's trajectory visualized as a continuous line.
        \item A legend showing the start and goal positions.
    \end{itemize}
    \item exception: Raises \texttt{PLOT\_ERROR} if path data is empty or not formatted correctly.
\end{itemize}

\noindent \texttt{plotMetrics}$(metrics)$:
\begin{itemize}
    \item transition: Reads performance metrics (e.g., execution time, path length) and generates a bar chart or line graph.
    \item output: Displays a graphical comparison of metrics with labeled axes.
    \item exception: Raises \texttt{PLOT\_ERROR} if metric data is missing or improperly formatted.
\end{itemize}

\subsubsection{Local Functions}

The following helper functions are used internally for plotting:

\noindent \texttt{drawTrajectory}(path):
\begin{itemize}
    \item Reads the sequence of waypoints and plots them on a 2D plane.
\end{itemize}

\noindent \texttt{addAnnotations}(plot):
\begin{itemize}
    \item Adds titles, labels, and legends to the generated plot.
\end{itemize}

\newpage



\newpage

\section{MIS of Control Module} \label{Module:Control}

\subsection{Module}

Control Module

\subsection{Uses}

This module interacts with:
\begin{itemize}
    \item \textbf{Input Parameters Module} (to retrieve initial configuration, goal configuration, robot parameters, and obstacle definitions)
    \item \textbf{Path Planning Module} (to generate a collision-free path)
    \item \textbf{Collision Detection Module} (to verify the safety of planned paths)
    \item \textbf{Inverse Kinematics Solver Module} (to compute joint configurations along the path)
    \item \textbf{Output Verification Module} (to validate computed outputs)
    \item \textbf{Plotting Module} (to visualize the planned path)
    \item \textbf{Logging and Debugging Module} (to log execution details and debug information)
\end{itemize}

\subsection{Syntax}

\subsubsection{Exported Constants}

None.

\subsubsection{Exported Access Programs}

\begin{center}
\renewcommand{\arraystretch}{1.4}
\begin{tabular}{p{3.5cm} p{5cm} p{3cm} p{3cm}}
\toprule
\textbf{Name} & \textbf{In} & \textbf{Out} & \textbf{Exceptions} \\
\midrule
\texttt{executePlanning} & - & ADT(Path) & CONTROL\_ERR \\[4pt]
\bottomrule
\end{tabular}
\end{center}

\subsection{Semantics}

\subsubsection{State Variables}

None. This module does not maintain any persistent state.

\subsubsection{Environment Variables}

None. This module does not interact with external devices.

\subsubsection{Assumptions}

\begin{itemize}
    \item All input parameters provided by the Input Parameters Module are valid and complete.
    \item The modules used by this control module operate without internal errors under normal circumstances.
    \item No dynamic changes occur in the environment during the planning and verification process.
\end{itemize}

\subsubsection{Access Routine Semantics}

\noindent \texttt{executePlanning}():
\begin{itemize}
    \item transition: The Control Module orchestrates the entire path planning process, executing the following sequential operations:
    
    \begin{enumerate}
        \item Retrieves initial and goal configurations, robot parameters, and obstacle data from the Input Parameters Module.
        \item Invokes the Path Planning Module to compute an initial collision-free path.
        \item Utilizes the Collision Detection Module to ensure the generated path is collision-free.
        \item Calls the Inverse Kinematics Solver Module to compute feasible joint angles corresponding to the planned path.
        \item Employs the Output Verification Module to validate the correctness and feasibility of the computed path.
        \item Passes the verified path to the Plotting Module for visualization.
        \item Records relevant information and any encountered issues via the Logging and Debugging Module.
    \end{enumerate}

    \item output: Returns a verified and collision-free path (ADT(Path)), consisting of a valid sequence of robot configurations.
    
    \item exception: Raises \texttt{CONTROL\_ERR} if any step in the planning or verification sequence fails, or if internal communication with any module is disrupted.
\end{itemize}

\subsubsection{Local Functions}

None.




\newpage
















\bibliographystyle {plainnat}
\bibliography {../../../refs/References}

\newpage




\section{Appendix} \label{Appendix}

\wss{Extra information if required}

\newpage{}

\section*{Appendix --- Reflection}

\wss{Not required for CAS 741 projects}

The information in this section will be used to evaluate the team members on the
graduate attribute of Problem Analysis and Design.

The purpose of reflection questions is to give you a chance to assess your own
learning and that of your group as a whole, and to find ways to improve in the
future. Reflection is an important part of the learning process.  Reflection is
also an essential component of a successful software development process.  

Reflections are most interesting and useful when they're honest, even if the
stories they tell are imperfect. You will be marked based on your depth of
thought and analysis, and not based on the content of the reflections
themselves. Thus, for full marks we encourage you to answer openly and honestly
and to avoid simply writing ``what you think the evaluator wants to hear.''

Please answer the following questions.  Some questions can be answered on the
team level, but where appropriate, each team member should write their own
response:


\begin{enumerate}
  \item What went well while writing this deliverable? 
  \item What pain points did you experience during this deliverable, and how
    did you resolve them?
  \item Which of your design decisions stemmed from speaking to your client(s)
  or a proxy (e.g. your peers, stakeholders, potential users)? For those that
  were not, why, and where did they come from?
  \item While creating the design doc, what parts of your other documents (e.g.
  requirements, hazard analysis, etc), it any, needed to be changed, and why?
  \item What are the limitations of your solution?  Put another way, given
  unlimited resources, what could you do to make the project better? (LO\_ProbSolutions)
  \item Give a brief overview of other design solutions you considered.  What
  are the benefits and tradeoffs of those other designs compared with the chosen
  design?  From all the potential options, why did you select the documented design?
  (LO\_Explores)
\end{enumerate}


\end{document}