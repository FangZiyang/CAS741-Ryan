\documentclass[12pt, titlepage]{article}

\usepackage{amsmath, mathtools}

\usepackage[round]{natbib}
\usepackage{amsfonts}
\usepackage{amssymb}
\usepackage{graphicx}
\usepackage{colortbl}
\usepackage{xr}
\usepackage{hyperref}
\usepackage{longtable}
\usepackage{xfrac}
\usepackage{tabularx}
\usepackage{float}
\usepackage{siunitx}
\usepackage{booktabs}
\usepackage{multirow}
\usepackage[section]{placeins}
\usepackage{caption}
\usepackage{fullpage}

\hypersetup{
bookmarks=true,     % show bookmarks bar?
colorlinks=true,       % false: boxed links; true: colored links
linkcolor=red,          % color of internal links (change box color with linkbordercolor)
citecolor=blue,      % color of links to bibliography
filecolor=magenta,  % color of file links
urlcolor=cyan          % color of external links
}

\usepackage{array}

\externaldocument{../../SRS/SRS}

\input{../../Comments}
%% Common Parts

\newcommand{\progname}{2D-RAPP} % PUT YOUR PROGRAM NAME HERE
\newcommand{\authname}{Ziyang(Ryan) Fang} % AUTHOR NAMES                  

\usepackage{hyperref}
    \hypersetup{colorlinks=true, linkcolor=blue, citecolor=blue, filecolor=blue,
                urlcolor=blue, unicode=false}
    \urlstyle{same}
                                


\begin{document}

\title{Module Interface Specification for \progname{}}

\author{\authname}

\date{\today}

\maketitle

\pagenumbering{roman}

\section{Revision History}

\begin{tabularx}{\textwidth}{p{3cm}p{2cm}X}
\toprule {\bf Date} & {\bf Version} & {\bf Notes}\\
\midrule
2025 March 19 & 1.0 & Notes\\
\bottomrule
\end{tabularx}

~\newpage

\section{Symbols, Abbreviations and Acronyms}

See SRS Documentation at \url{https://github.com/FangZiyang/CAS741-Ryan/blob/main/docs/SRS/SRS.pdf}.


\newpage

\tableofcontents

\newpage

\pagenumbering{arabic}

\section{Introduction}


The following document details the Module Interface Specifications for
\textbf{2D Robot Arm Path Planning} 

Complementary documents include the System Requirement Specifications
and Module Guide.  The full documentation and implementation can be
found at \url{https://github.com/FangZiyang/CAS741-Ryan}. 

% This document is the Module Interface Specification (MIS) for the system. It is developed in accordance with the Software Requirements Specification (SRS) for the project. The MIS defines the modular structure of the software and the interface of each module that together fulfill the requirements for planning a collision-free path for a two-dimensional robotic arm. The system's goal, as outlined in the SRS, is to compute a safe and efficient path for the robotic arm to move from a given starting position to a desired goal position without colliding with any obstacles in its environment.

% The design described herein is implemented in \textbf{Python}, aligning with the SRS implementation constraints. Each module's responsibilities and interactions are clearly delineated to ensure the system meets the functional goals (path computation, collision avoidance, etc.) and outputs specified in the SRS. A pathfinding algorithm (such as \textit{A*}) is utilized within the Path Planning module to find an optimal route; however, the detailed mechanics of this algorithm are abstracted away in this document, focusing instead on the interface and role of the module within the system.


\section{Notation}
The structure of the MIS for modules comes from \citet{HoffmanAndStrooper1995},
with the addition that template modules have been adapted from
\cite{GhezziEtAl2003}.  The mathematical notation comes from Chapter 3 of
\citet{HoffmanAndStrooper1995}.  For instance, the symbol := is used for a
multiple assignment statement and conditional rules follow the form $(c_1
\Rightarrow r_1 | c_2 \Rightarrow r_2 | ... | c_n \Rightarrow r_n )$.

The following table summarizes the primitive data types used by \progname. 

\begin{center}
  \renewcommand{\arraystretch}{1.2}
  \noindent 
  \begin{tabular}{l l p{7.5cm}} 
  \toprule 
  \textbf{Data Type} & \textbf{Notation} & \textbf{Description}\\ 
  \midrule
  character & char & a single symbol or digit\\
  integer & $\mathbb{Z}$ & a number without a fractional component in $(-\infty, \infty)$ \\[4pt]
  natural number & $\mathbb{N}$ & a number without a fractional component in $[1, \infty)$ \\[4pt]
  real & $\mathbb{R}$ & any number in $(-\infty, \infty)$\\[4pt]
  point & $\mathbb{R} \times \mathbb{R}$ & ordered pair representing a 2D coordinate\\[4pt]
  configuration & $\mathbb{R}^n$ & vector of joint angles, length $n$ corresponds to the number of joints\\[4pt]
  path & sequence of configurations & ordered list of configurations representing a robot path\\[4pt]
  obstacle & (point, real) & tuple containing position (center) and radius defining a circular obstacle\\
  \bottomrule
  \end{tabular} 
  \end{center}

\noindent
The specification of \progname \ uses some derived data types: sequences, strings, and
tuples. Sequences are lists filled with elements of the same data type. Strings
are sequences of characters. Tuples contain a list of values, potentially of
different types. In addition, \progname \ uses functions, which
are defined by the data types of their inputs and outputs. Local functions are
described by giving their type signature followed by their specification.

\section{Module Decomposition}

The following table is taken directly from the Module Guide document for this project.


\begin{table}[h!]
  \centering
  \begin{tabular}{p{0.3\textwidth} p{0.6\textwidth}}
  \toprule
  \textbf{Level 1} & \textbf{Level 2}\\
  \midrule
  \textbf{Hardware-Hiding Module} & Simulation Interface Module\\
  \midrule
  \multirow{4}{0.3\textwidth}{\textbf{Behaviour-Hiding Module}} 
  & Input Parameters Module \\
  & Output Format Module \\
  & Output Verification Module \\
  & Path Planning Module \\
  & Collision Detection Module \\
  & Inverse Kinematics Solver Module \\
  & Configuration Management Module \\
  \midrule
  \multirow{3}{0.3\textwidth}{\textbf{Software Decision Module}} 
  & Plotting Module \\
  & Logging and Debugging Module \\
  \bottomrule
  \end{tabular}
  \caption{Module Hierarchy}
  \label{TblMH}
  \end{table}


\newpage
~\newpage

\section{MIS of Path Planning Module} \label{Module:PathPlanning}

\subsection{Module}
Path Plann

\subsection{Uses}
This module interacts with the following components:
\begin{itemize}
\item \textbf{Input Parameters Module} (for start and goal configurations, obstacles, and robot parameters)
\item \textbf{Collision Avoidance Module} (to verify that planned paths are free of collisions)
\item \textbf{Output Format Module} (to return and display the final planned path)
\item \textbf{Data Structures Module} (for managing robot configurations, obstacles, and paths)
\end{itemize}


\subsection{Syntax}

\subsubsection{Exported Constants}

\begin{itemize}
  \item \texttt{MAX\_ITERATIONS} : $\mathbb{N}$ \quad (Maximum iterations allowed for path search, default: 10,000)
  \item \texttt{HEURISTIC\_FACTOR} : $\mathbb{R}$ \quad (Scaling factor for heuristic function in A* search, default: 1.0)
\end{itemize}


\subsubsection{Exported Access Programs}

\begin{center}
  \renewcommand{\arraystretch}{1.4}
  \begin{tabular}{p{3.5cm} p{3cm} p{4cm} p{4cm}}
  \toprule
  \textbf{Name} & \textbf{In} & \textbf{Out} & \textbf{Exceptions} \\
  \midrule
  \texttt{load\_params} & string & - & FileNotFound, InvalidFormat, MissingParameter, InvalidValue \\[4pt]
  
  \texttt{verify\_params} & - & - & InvalidRobotDimensions, InvalidJointLimits, InvalidObstacleData, InvalidInitialPose, InvalidGoalPose \\[4pt]
  
  \texttt{getRobotParams} & - & RobotParams & - \\[4pt]
  
  \texttt{getObstacles} & - & Obstacles & - \\[4pt]
  
  \texttt{getInitialConfig} & - & $\mathbb{R}^n$ & - \\[4pt]
  
  \texttt{getGoalConfig} & - & $\mathbb{R}^n$ & - \\[4pt]
  
  \texttt{planPath} & $\mathbb{R}^n \times \mathbb{R}^n \times Obstacles \times RobotParams$ & sequence of $\mathbb{R}^n$ & NoPathFoundException \\
  
  \bottomrule
  \end{tabular}
  \end{center}

  

\subsection{Semantics}

\subsubsection{State Variables}

None. The module does not maintain any persistent state.

\subsubsection{Environment Variables}

None. The module does not interact with external devices.

\subsubsection{Assumptions}


\begin{itemize}
  \item The robotic arm operates in a 2D plane.
  \item The environment is known beforehand and does not change dynamically.
  \item Obstacles are represented as circles with given positions and radii.
  \item The A* algorithm is used for path search in joint space.
  \end{itemize}

  \subsubsection{Access Routine Semantics}

  \noindent \texttt{load\_params}(filePath: string):
  \begin{itemize}
  \item transition: Parameters (robot dimensions, joint limits, initial and goal configurations, obstacles) are loaded from the given file into the module's state variables.
  \item exception: Raises \texttt{FileNotFound} if the file does not exist; raises \texttt{InvalidFormat}, \texttt{MissingParameter}, or \texttt{InvalidValue} if any input is invalid.
  \end{itemize}
  
  \noindent \texttt{verify\_params}():
  \begin{itemize}
  \item output: none
  \item exception: Raises \texttt{InvalidRobotDimensions}, \texttt{InvalidJointLimits}, \texttt{InvalidObstacleData}, \texttt{InvalidInitialPose}, or \texttt{InvalidGoalPose} if the loaded parameters violate constraints or logical consistency.
  \end{itemize}
  
  \noindent \texttt{getRobotParams}():
  \begin{itemize}
  \item output: \( RobotParams \), a tuple containing robot dimension parameters and joint constraints.
  \item exception: none
  \end{itemize}
  
  \noindent \texttt{getObstacles}():
  \begin{itemize}
  \item output: A sequence of \( Obstacles \), where each obstacle is represented as a tuple \((\mathbb{R}^2, \mathbb{R})\), containing the center coordinates and radius.
  \item exception: none
  \end{itemize}
  
  \noindent \texttt{getInitialConfig}():
  \begin{itemize}
  \item output: Initial robot configuration as a vector in \( \mathbb{R}^n \).
  \item exception: none
  \end{itemize}
  
  \noindent \texttt{getGoalConfig}():
  \begin{itemize}
  \item output: Goal robot configuration as a vector in \( \mathbb{R}^n \).
  \item exception: none
  \end{itemize}
  
  \noindent \texttt{planPath}(start: \( \mathbb{R}^n \), goal: \( \mathbb{R}^n \), obstacles: \( Obstacles \), robotParams: \( RobotParams \)):
  \begin{itemize}
  \item output: A collision-free sequence of robot configurations (path) from the initial configuration (\(start\)) to the goal configuration (\(goal\)), represented as a sequence of \( \mathbb{R}^n \).
  \item exception: Raises \texttt{NoPathFoundException} if no collision-free path exists between the provided configurations given the environment and robot constraints.
  \end{itemize}
  

  \subsubsection{Local Functions}

  The following local functions are defined for specifying internal logic clearly. These functions are primarily for specification purposes; they may not necessarily correspond directly to explicit implementations.
  
  \noindent \texttt{isValidConfig}(config: $\mathbb{R}^n$, robotParams: $RobotParams$):
  \begin{itemize}
    \item output: boolean, true if the given robot configuration satisfies the joint angle constraints, false otherwise.
  \end{itemize}
  
  \noindent \texttt{isCollisionFree}(config: $\mathbb{R}^n$, obstacles: $Obstacles$, robotParams: $RobotParams$):
  \begin{itemize}
    \item output: boolean, true if the given configuration does not result in collisions with obstacles, false otherwise.
  \end{itemize}
  
  \noindent \texttt{heuristicCost}(config1: $\mathbb{R}^n$, config2: $\mathbb{R}^n$):
  \begin{itemize}
    \item output: real, heuristic cost (estimated distance or effort) between two configurations for guiding the A* algorithm.
  \end{itemize}
  
  \noindent \texttt{distance}(config1: $\mathbb{R}^n$, config2: $\mathbb{R}^n$):
  \begin{itemize}
    \item output: real, actual distance metric used to compute the edge costs between configurations during path planning.
  \end{itemize}
  

  \newpage
  \newpage  





\section{MIS of Collision Detection Module} \label{Module:CollisionDetection}

\subsection{Module}
Collision Detection

\subsection{Uses}
This module interacts with:
\begin{itemize}
\item \textbf{Input Parameters Module} (to obtain obstacle definitions and robot dimensions)
\item \textbf{Path Planning Module} (to provide collision checks for planned paths)
\item \textbf{Inverse Kinematics Solver Module} (to verify joint configurations are collision-free)
\end{itemize}

\subsection{Syntax}

\subsubsection{Exported Constants}
None.

\subsubsection{Exported Access Programs}

\begin{center}
  \renewcommand{\arraystretch}{1.4}
  \begin{tabular}{p{4cm} p{4cm} p{3cm} p{4cm}}
    \toprule
    \textbf{Name} & \textbf{In} & \textbf{Out} & \textbf{Exceptions} \\
    \midrule
    \texttt{checkCollision} & $\mathbb{R}^n \times Obstacles \times RobotParams$ & boolean & InvalidConfiguration \\[4pt]
    \texttt{checkPathCollision} & sequence of $\mathbb{R}^n$ $\times$ Obstacles $\times$ RobotParams & boolean & InvalidPath \\[4pt]
    \bottomrule
  \end{tabular}
\end{center}

\subsection{Semantics}

\subsubsection{State Variables}
None. This module does not maintain any persistent state.

\subsubsection{Environment Variables}
None. This module does not interact with external devices or environments.

\subsubsection{Assumptions}

\begin{itemize}
  \item Obstacles are defined as circles with a given position and radius.
  \item Robot arm links are modeled as line segments between joint positions.
  \item The environment and obstacle positions remain static during collision detection.
\end{itemize}

\subsubsection{Access Routine Semantics}

\noindent \texttt{checkCollision}(config: $\mathbb{R}^n$, obstacles: Obstacles, robotParams: RobotParams):
\begin{itemize}
\item output: Returns \texttt{true} if the provided robot configuration collides with any obstacle; otherwise, returns \texttt{false}.
\item exception: Raises \texttt{InvalidConfiguration} if the configuration provided violates joint constraints or robot parameters.
\end{itemize}

\noindent \texttt{checkPathCollision}(path: sequence of $\mathbb{R}^n$, obstacles: Obstacles, robotParams: RobotParams):
\begin{itemize}
\item output: Returns \texttt{true} if any configuration in the provided path results in a collision; otherwise, returns \texttt{false}.
\item exception: Raises \texttt{InvalidPath} if the provided path contains invalid configurations or is improperly defined.
\end{itemize}

\subsubsection{Local Functions}

The following functions are local specification tools that clarify the module's collision-checking logic:

\noindent \texttt{detectSegmentCollision}(pointA: $\mathbb{R}^2$, pointB: $\mathbb{R}^2$, obstacle: ( $\mathbb{R}^2$, $\mathbb{R}$ )):
\begin{itemize}
  \item output: boolean, true if the line segment defined by \texttt{pointA} and \texttt{pointB} intersects the obstacle; false otherwise.
\end{itemize}

\noindent \texttt{forwardKinematics}(config: $\mathbb{R}^n$, robotParams: RobotParams):
\begin{itemize}
  \item output: sequence of $\mathbb{R}^2$, representing the positions of all joints and the end-effector, used for collision checking.
\end{itemize}


\newpage

\bibliographystyle {plainnat}
\bibliography {../../../refs/References}

\newpage




\section{Appendix} \label{Appendix}

\wss{Extra information if required}

\newpage{}

\section*{Appendix --- Reflection}

\wss{Not required for CAS 741 projects}

The information in this section will be used to evaluate the team members on the
graduate attribute of Problem Analysis and Design.

\input{../../Reflection.tex}

\begin{enumerate}
  \item What went well while writing this deliverable? 
  \item What pain points did you experience during this deliverable, and how
    did you resolve them?
  \item Which of your design decisions stemmed from speaking to your client(s)
  or a proxy (e.g. your peers, stakeholders, potential users)? For those that
  were not, why, and where did they come from?
  \item While creating the design doc, what parts of your other documents (e.g.
  requirements, hazard analysis, etc), it any, needed to be changed, and why?
  \item What are the limitations of your solution?  Put another way, given
  unlimited resources, what could you do to make the project better? (LO\_ProbSolutions)
  \item Give a brief overview of other design solutions you considered.  What
  are the benefits and tradeoffs of those other designs compared with the chosen
  design?  From all the potential options, why did you select the documented design?
  (LO\_Explores)
\end{enumerate}


\end{document}