\documentclass[12pt, titlepage]{article}

\usepackage{amsmath, mathtools}

\usepackage[round]{natbib}
\usepackage{amsfonts}
\usepackage{amssymb}
\usepackage{graphicx}
\usepackage{colortbl}
\usepackage{xr}
\usepackage{hyperref}
\usepackage{longtable}
\usepackage{xfrac}
\usepackage{tabularx}
\usepackage{float}
\usepackage{siunitx}
\usepackage{booktabs}
\usepackage{multirow}
\usepackage[section]{placeins}
\usepackage{caption}
\usepackage{fullpage}


\hypersetup{
bookmarks=true,     % show bookmarks bar?
colorlinks=true,       % false: boxed links; true: colored links
linkcolor=red,          % color of internal links (change box color with linkbordercolor)
citecolor=blue,      % color of links to bibliography
filecolor=magenta,  % color of file links
urlcolor=cyan          % color of external links
}

\usepackage{array}

\externaldocument{../../SRS/SRS}

\input{../../Comments}
%% Common Parts

\newcommand{\progname}{2D-RAPP} % PUT YOUR PROGRAM NAME HERE
\newcommand{\authname}{Ziyang(Ryan) Fang} % AUTHOR NAMES                  

\usepackage{hyperref}
    \hypersetup{colorlinks=true, linkcolor=blue, citecolor=blue, filecolor=blue,
                urlcolor=blue, unicode=false}
    \urlstyle{same}
                                


\begin{document}

\title{Module Interface Specification for \progname{}}

\author{\authname}

\date{\today}

\maketitle

\pagenumbering{roman}

\section{Revision History}

\begin{tabularx}{\textwidth}{p{3cm}p{2cm}X}
\toprule {\bf Date} & {\bf Version} & {\bf Notes}\\
\midrule
March 19 2025 & 1.0 & Notes\\
April 05 2025 & 2.0 & Notes\\
\bottomrule
\end{tabularx}

~\newpage

\section{Symbols, Abbreviations and Acronyms}

See SRS Documentation at \url{https://github.com/FangZiyang/CAS741-Ryan/blob/main/docs/SRS/SRS.pdf}.


\newpage

\tableofcontents

\newpage

\pagenumbering{arabic}

\section{Introduction}


The following document details the Module Interface Specifications for
\textbf{2D Robot Arm Path Planning} 

Complementary documents include the System Requirement Specifications
and Module Guide.  The full documentation and implementation can be
found at \url{https://github.com/FangZiyang/CAS741-Ryan}. 




\section{Notation}
The structure of the MIS for modules comes from \citet{HoffmanAndStrooper1995},
with the addition that template modules have been adapted from
\cite{GhezziEtAl2003}.  The mathematical notation comes from Chapter 3 of
\citet{HoffmanAndStrooper1995}.  For instance, the symbol := is used for a
multiple assignment statement and conditional rules follow the form $(c_1
\Rightarrow r_1 | c_2 \Rightarrow r_2 | ... | c_n \Rightarrow r_n )$.

The following table summarizes the primitive data types used by \progname. 

\begin{center}
  \renewcommand{\arraystretch}{1.2}
  \noindent 
  \begin{tabular}{l l p{7.5cm}} 
  \toprule 
  \textbf{Data Type} & \textbf{Notation} & \textbf{Description}\\ 
  \midrule
  character & char & a single symbol or digit\\
  integer & $\mathbb{Z}$ & a number without a fractional component in $(-\infty, \infty)$ \\[4pt]
  natural number & $\mathbb{N}$ & a number without a fractional component in $[1, \infty)$ \\[4pt]
  real & $\mathbb{R}$ & any number in $(-\infty, \infty)$\\[4pt]

  \bottomrule
  \end{tabular} 
  \end{center}

\noindent
The specification of \progname \ uses some derived data types: sequences, strings, and
tuples. Sequences are lists filled with elements of the same data type. Strings
are sequences of characters. Tuples contain a list of values, potentially of
different types. In addition, \progname \ uses functions, which
are defined by the data types of their inputs and outputs. Local functions are
described by giving their type signature followed by their specification.

\section{Module Decomposition}

The following table is taken directly from the Module Guide document for this project.


\begin{table}[h!]
  \centering
  \begin{tabular}{p{0.3\textwidth} p{0.6\textwidth}}
  \toprule
  \textbf{Level 1} & \textbf{Level 2}\\
  \midrule
  \textbf{Hardware Hiding} & \\ 
  \midrule
  \multirow{6}{0.3\textwidth}{\textbf{Behaviour Hiding}} 
  & Input Parameters Module \\ 
  & Output Format Module \\ 
  & Output Verification Module \\ 
  & Inverse Kinematics Solver Module \\ 
  & Configuration Management Module \\ 
  & Path Planning Module \\ 
  & Collision Detection Module \\ 
  & Controll Module\\
  & Data Types Module\\
  \midrule
  \multirow{1}{0.3\textwidth}{\textbf{Software Decision}} 
  & Plotting Module \\ 
  \bottomrule
  \end{tabular}
  \caption{Module Hierarchy}
  \label{TblMH}
\end{table}


\newpage
~\newpage


%=====================================================================
% 6  MIS of Path‑Planning Module
%=====================================================================
\section{MIS of Path‑Planning Module}
\label{mod:pathplanning}

%---------------------------------------------------------------------
\subsection{Module}
Path‑Planning

%---------------------------------------------------------------------
\subsection{Uses}
\begin{itemize}
  \item \textbf{Input Parameters Module} (provides start/goal configurations,\\
        \hspace*{1.8em}\;obstacles, robot parameters)
  \item \textbf{Collision Detection Module} (verifies that candidate paths are collision–free)
  \item \textbf{Output Format Module} (formats the sequence of configurations for visualisation)
  \item \textbf{Data‑Types Module} 
\end{itemize}

%---------------------------------------------------------------------
\subsection{Syntax}
%.....................................................................


%.....................................................................
\subsubsection{Exported Constants}
\begin{description}
  \item[\texttt{MAX\_ITER}] $\!:\! \mathbb{N}$ \hfill (default 10\,000)
  \item[\texttt{HEURISTIC\_K}] $\!:\! \mathbb{R}$ \hfill (default $1.0$)
\end{description}

%.....................................................................
\subsubsection{Exported Access Programs}
\begin{center}
\renewcommand{\arraystretch}{1.35}
\begin{tabular}{p{3.1cm} p{5.3cm} p{3.8cm} p{3.2cm}}
\toprule
\textbf{Name} & \textbf{In} & \textbf{Out} & \textbf{Exceptions}\\ \midrule
\texttt{planPath} & \texttt{Config} $\times$ \texttt{Config} $\times$ \texttt{Obstacles} $\times$ \texttt{RobotParams} & \texttt{Path} & \texttt{PathNotFound} \\
\bottomrule
\end{tabular}
\end{center}

%---------------------------------------------------------------------
\subsection{Semantics}
%.....................................................................
\subsubsection{State Variables}
None.

%.....................................................................
\subsubsection{Assumptions}
\begin{itemize}
  \item The module receives complete and valid \textit{start}, \textit{goal}, \textit{obstacles}
        and \textit{robot parameters} from the Input‑Parameters Module.
  \item Joint‑space is treated as an $n$‑dimensional torus
        (wrap‑around at $360^{\circ}$).
\end{itemize}

%.....................................................................
\subsubsection{Access Routine Semantics}

\paragraph{\texttt{planPath(start, goal, obstacles, robotParams)}}
\begin{itemize}
  \item \textbf{output:} A feasible path $p = \langle q_0,\dots,q_m\rangle$ such that:
    \begin{itemize}
      \item $q_0 = start$
      \item $q_m = goal$
      \item $\forall i \in [0, m-1]$: $q_{i+1} \in succ(q_i)$ and \\ $\neg$checkCollision$(q_i,\textit{obstacles},\textit{robotParams})$
      \item The path is generated using A* with cost function: \\
        $f(q_i) = gCost(q_0, q_i) + hCost(q_i)$
    \end{itemize}
  \item \textbf{exception:} \texttt{PathNotFound} if no feasible path is found within \texttt{MAX\_ITER}.
\end{itemize}

%.....................................................................
\subsubsection{Local Functions}
\begin{description}
  \item[\texttt{succ}($q$)]  returns the neighbouring configurations of $q$ according to the lattice resolution.
  \item[\texttt{gCost}($q_i,q_{i+1}$)] $\;=\; \|q_{i+1}-q_i\|_2$
  \item[\texttt{hCost}($q$)] $\;=\;$ \texttt{HEURISTIC\_K}$\:\|q-\textit{goal}\|_2$
\end{description}

\newpage




%=====================================================================
% 7  MIS of Collision‑Detection Module
%=====================================================================


\section{MIS of Collision‑Detection Module}
\label{mod:collision}

%---------------------------------------------------------------------
\subsection{Module}
Collision‑Detection

%---------------------------------------------------------------------
\subsection{Uses}
\begin{itemize}
  \item \textbf{Input Parameters Module} (provides \textit{obstacles} and \textit{robot parameters})
\end{itemize}

%---------------------------------------------------------------------
\subsection{Syntax}
%.....................................................................
\subsubsection{Exported Types}
\begin{description}
  \item[\texttt{Config}] (as defined in \S\ref{mod:pathplanning})
\end{description}

%.....................................................................
\subsubsection{Exported Access Programs}
\begin{center}
\renewcommand{\arraystretch}{1.35}
\begin{tabular}{p{4.1cm} p{5.3cm} p{3.4cm} p{3.1cm}}
  \toprule
  \textbf{Name} & \textbf{In} & \textbf{Out} & \textbf{Exceptions}\\ \midrule
  \texttt{checkCollision} & \texttt{Config} $\times$ \texttt{Obstacles} $\times$ \texttt{RobotParams} & boolean & \texttt{InvalidConfig}\\
  \bottomrule
  \end{tabular}
  
\end{center}

%---------------------------------------------------------------------
\subsection{Semantics}
%.....................................................................
\subsubsection{State Variables}
None.

%.....................................................................
\subsubsection{Assumptions}
None

%.....................................................................
\subsubsection{Access Routine Semantics}

\paragraph{\texttt{checkCollision}}
\begin{itemize}
  \item \textbf{description:} Let $P=\langle p_0,\dots,p_n\rangle$ be the joint positions obtained from \texttt{forwardKinematics}$(config,robotParams)$.  
        A link $\overline{p_{i-1}p_i}$ collides with obstacle $O=\langle c,r\rangle$ iff
        \[
          d(\overline{p_{i-1}p_i},c) \le r
          \quad\text{where}\quad
          d = \frac{|(x_2-x_1)(y_1-y_c)-(y_2-y_1)(x_1-x_c)|}
                 {\sqrt{(x_2-x_1)^2+(y_2-y_1)^2}}
        \]
  \item \textbf{output:} \texttt{true} iff any link collides with any obstacle.
  \item \textbf{exception:} \texttt{InvalidConfig} if joint limits are violated.
\end{itemize}

% \paragraph{\texttt{checkPathCollision}}
% \begin{itemize}
%   \item \textbf{description:} Returns \texttt{true} iff
%         $\exists q\in\textit{path}:\texttt{checkCollision}(q,\textit{obstacles},\textit{robotParams})$
%   \item \textbf{output:} boolean as above.
%   \item \textbf{exception:} \texttt{InvalidPath} if \textit{path} is empty or contains an \texttt{InvalidConfig}.
% \end{itemize}

%.....................................................................
\subsubsection{Local Functions}
\begin{description}
  \item[\texttt{checkPathCollision}] returns \texttt{true} if any configuration in the path collides with an obstacle.
  \item[\texttt{forwardKinematics}] $\!:\! \texttt{Config} \times \texttt{RobotParams} \rightarrow$ sequence of $\mathbb{R}^2$
\end{description}

\newpage



%=====================================================================
% 9  MIS of Inverse‑Kinematics‑Solver Module
%=====================================================================
\section{MIS of Inverse‑Kinematics Solver Module}
\label{mod:ik}

\subsection{Module}
Inverse‑Kinematics Solver

\subsection{Uses}
\begin{itemize}
  \item \textbf{Input Parameters Module}
  \item \textbf{Collision Detection Module}
\end{itemize}

%---------------------------------------------------------------------
\subsection{Syntax}
\subsubsection{Exported Access Programs}
\begin{center}
\renewcommand{\arraystretch}{1.3}
\begin{tabular}{p{3.2cm} p{6cm} p{3cm} p{3cm}}
\toprule
\textbf{Name} & \textbf{In} & \textbf{Out} & \textbf{Exceptions}\\ \midrule
\texttt{solveIK} & \texttt{Point} $\times$ \texttt{RobotParams} & sequence of \texttt{Config} & \texttt{NoSolution}\\
\bottomrule
\end{tabular}
\end{center}

%---------------------------------------------------------------------
\subsection{Semantics}
\subsubsection{State Variables}
None.

\subsubsection{Assumptions}
\begin{itemize}
  \item The forward‑kinematics map $FK:\texttt{Config}\rightarrow\texttt{Point}$ is continuous and differentiable.
  \item The target point lies inside the reachable workspace.
\end{itemize}

\subsubsection{Access‑Routine Semantics}

\paragraph{\texttt{solveIK}(Point, RobotParams)}
\begin{itemize}
  \item \textbf{output:} A finite sequence $S=\langle q_0,\dots,q_k\rangle$ of joint configurations such that
        $\|FK(q_i)-\textit{target}\|\le \varepsilon$ for every $q_i\in S$.
  \item \textbf{exception:} \texttt{NoSolution} raised if the iterative algorithm exceeds
        \texttt{MAX\_ITER} without satisfying the tolerance.
\end{itemize}


%=====================================================================
% 10  MIS of Output‑Verification Module
%=====================================================================
\newpage
\section{MIS of Output‑Verification Module}
\label{mod:verify}

\subsection{Module}
Output‑Verification

\subsection{Uses}
\begin{itemize}
  \item \textbf{Collision Detection Module}
\end{itemize}

%---------------------------------------------------------------------
\subsection{Syntax}
\subsubsection{Exported Constant}
\texttt{TOL\_ERR} $=10^{-6}$

\subsubsection{Exported Access Programs}
\begin{center}
\renewcommand{\arraystretch}{1.3}
\begin{tabular}{p{3cm} p{6.5cm} p{3cm} p{3cm}}
\toprule
\textbf{Name} & \textbf{In} & \textbf{Out} & \textbf{Exceptions}\\ \midrule
\texttt{verifyPath} & \texttt{Path} $\times$ \texttt{Obstacles} $\times$ \texttt{RobotParams} & --- & \texttt{PathInvalid}, \texttt{Collision}\\
\bottomrule
\end{tabular}
\end{center}

%---------------------------------------------------------------------
\subsection{Semantics}
\subsubsection{State Variables}
None.

\subsubsection{Assumptions}
The supplied path has already been discretised into configurations of type \texttt{Config}.

\subsubsection{Access‑Routine Semantics}

\paragraph{\texttt{verifyPath}}
\begin{itemize}
  \item \textbf{description:}
        \begin{enumerate}
          \item Check joint‑limit and self‑collision constraints for every $q\in\texttt{path}$.
          \item Call \texttt{checkCollision} from the Collision‑Detection Module.
        \end{enumerate}
  \item \textbf{exception:}
        \begin{itemize}
          \item \texttt{PathInvalid} if any configuration violates step 1.
          \item \texttt{Collision} if step 2 reports a collision.
        \end{itemize}
\end{itemize}

%  (no local functions: all checks are delegated)

%=====================================================================
% 11  MIS of Plotting Module
%=====================================================================
\newpage
\section{MIS of Plotting Module}
\label{mod:plot}

\subsection{Module}
Plotting

\subsection{Uses}
None.

%---------------------------------------------------------------------
\subsection{Syntax}
\subsubsection{Exported Access Programs}
\begin{center}
\renewcommand{\arraystretch}{1.3}
\begin{tabular}{p{3.2cm} p{6cm} p{3cm} p{3cm}}
\toprule
\textbf{Name} & \textbf{In} & \textbf{Out} & \textbf{Exceptions}\\ \midrule
\texttt{plotPath}    & \texttt{Path} & --- & \texttt{PlotErr}\\
\texttt{plotMetrics} & \texttt{MetricTable} & --- & \texttt{PlotErr}\\
\bottomrule
\end{tabular}
\end{center}

%---------------------------------------------------------------------
\subsection{Semantics}
\subsubsection{State Variables}
None.

\subsubsection{Environment Variables}
\begin{itemize}
  \item \textit{win} : handle to the active 2‑D graphics window.
\end{itemize}

\subsubsection{Assumptions}
The graphics back‑end supports real‑time rendering.

\subsubsection{Access‑Routine Semantics}

\paragraph{\texttt{plotPath}}
\begin{itemize}
  \item \textbf{description:} Clears \textit{win} and draws:
        way‑points (circles), continuous trajectory (poly‑line),
        labels for start/goal.
  \item \textbf{exception:} \texttt{PlotErr} if \texttt{path} is empty.
\end{itemize}

\paragraph{\texttt{plotMetrics}}
\begin{itemize}
  \item \textbf{description:} Replaces the contents of \textit{win} with a bar‑
        or line‑chart of the supplied performance metrics.
  \item \textbf{exception:} \texttt{PlotErr} if the table is ill‑formed.
\end{itemize}



%=====================================================================
% 12  MIS of Control Module
%=====================================================================
\newpage
\section{MIS of Control Module}
\label{mod:control}

\subsection{Module}
Control

\subsection{Uses}
\begin{itemize}
  \item Input‑Parameters, Path‑Planning, Collision‑Detection,
        Inverse‑Kinematics, Output‑Verification, Plotting
\end{itemize}

%---------------------------------------------------------------------
\subsection{Syntax}
\subsubsection{Exported Access Programs}
\begin{center}
\renewcommand{\arraystretch}{1.3}
\begin{tabular}{p{3.2cm} p{4cm} p{3cm} p{3cm}}
\toprule
\textbf{Name} & \textbf{In} & \textbf{Out} & \textbf{Exceptions}\\ \midrule
\texttt{execute} & --- & \texttt{Path} & \texttt{CtrlErr}\\
\bottomrule
\end{tabular}
\end{center}

%---------------------------------------------------------------------
\subsection{Semantics}
\subsubsection{State Variables}
None.

\subsubsection{Assumptions}
All subordinate modules are already initialised.

\subsubsection{Access‑Routine Semantics}

\paragraph{\texttt{execute}}
\begin{itemize}
  \item \textbf{output:} Returns the verified, collision‑free
        \texttt{Path} $p$ produced by the following explicit calls:
        \begin{enumerate}
          \item $(\textit{init},\textit{goal},rParam,obs)\;=\;$\textbf{Input‑Parameters::}\texttt{getAll}()
          \item $p_0=\;$\textbf{Path‑Planning::}\texttt{planPath}$(\textit{init},\textit{goal},obs,rParam)$
          \item \textbf{Collision‑Detection::}\texttt{checkCollision}$(p_0,obs,rParam)$
          \item $p=\;$\textbf{Output‑Verification::}\texttt{verifyPath}$(p_0,obs,rParam)$
          \item \textbf{Plotting::}\texttt{plotPath}$(p)$
        \end{enumerate}
  \item \textbf{exception:} \texttt{CtrlErr} if any invoked access routine
        raises an exception.
\end{itemize}

% (no local functions)

\newpage



%=====================================================================
% 5  MIS of Data‑Types Module
%=====================================================================
\section{MIS of Data‑Types Module}
\label{mod:dtypes}

%---------------------------------------------------------------------
\subsection{Module}
Data‑Types

%---------------------------------------------------------------------
\subsection{Uses}
None.  This module is imported by other modules to share common
abstract data types (ADTs).

%---------------------------------------------------------------------
\subsection{Syntax}
\subsubsection{Exported Types}
\begin{description}
  %-------------------------------------------------- geometric / obstacle
  \item[\texttt{Point}] $\equiv \mathbb{R}^2$
        \hfill (Cartesian coordinate in the plane)
  \item[\texttt{ObstacleT}]
        $\equiv \langle c : \texttt{Point},\; r : \mathbb{R} \rangle$
  \item[\texttt{Obstacles}]
        $\equiv$ sequence of \texttt{ObstacleT}

  %-------------------------------------------------- robot parameters
  \item[\texttt{RobotParams}]
        $\equiv \langle
          L :$ sequence of $\mathbb{R},\;
          \textit{jointLim} :$ sequence of
          $\langle \ell : \mathbb{R},\, u : \mathbb{R}\rangle
        \rangle$

  %-------------------------------------------------- configuration / path
  \item[\texttt{Config}] $\equiv \mathbb{R}^n$
        \hfill (vector of joint angles)
  \item[\texttt{Path}] $\equiv$ sequence of \texttt{Config}

  %-------------------------------------------------- plotting / metrics
  \item[\texttt{MetricTable}]
        $\equiv$ set of key–value pairs
        $\langle \textit{name} : \texttt{string},\;
                 \textit{value} : \mathbb{R} \rangle$
\end{description}

\subsubsection{Exported Constants}
None.

\subsubsection{Exported Access Programs}
None.  The module only publishes type definitions.

%---------------------------------------------------------------------
\subsection{Semantics}
\subsubsection{State Variables}
None.

\subsubsection{Assumptions}
\begin{itemize}
  \item All numeric quantities are expressed in SI units
        (metres, radians, seconds) unless stated otherwise.
  \item The dimension~$n$ in \texttt{Config} equals the number
        of revolute joints in the robot and is fixed at run‑time
        by \texttt{RobotParams}.
\end{itemize}

\subsubsection{Access‑Routine Semantics}
Not applicable – no routines are exported.

\subsubsection{Local Functions}
None.


\newpage

%=====================================================================
% 5  MIS of Input‑Parameters Module
%=====================================================================
\section{MIS of Input‑Parameters Module}
\label{mod:input}

%---------------------------------------------------------------------
\subsection{Module}
Input‑Parameters

%---------------------------------------------------------------------
\subsection{Uses}
\begin{itemize}
  \item \textbf{Data‑Types Module} (for \texttt{Config}, \texttt{RobotParams}, \texttt{Obstacles})
\end{itemize}

%---------------------------------------------------------------------
\subsection{Syntax}
\subsubsection{Exported Types}
None.  All returned values use types defined in the Data‑Types module.

\subsubsection{Exported Constants}
None.

\subsubsection{Exported Access Programs}
\begin{center}
\renewcommand{\arraystretch}{1.35}
\begin{tabular}{p{3.2cm} p{5.8cm} p{3.4cm} p{3cm}}
\toprule
\textbf{Name} & \textbf{In} & \textbf{Out} & \textbf{Exceptions}\\ \midrule
\texttt{getAll} & --- & \texttt{Config} $\times$ \texttt{Config} $\times$ \texttt{RobotParams} $\times$ \texttt{Obstacles} & \texttt{ParamErr}\\
\bottomrule
\end{tabular}
\end{center}

%---------------------------------------------------------------------
\subsection{Semantics}
\subsubsection{State Variables}
\begin{itemize}
  \item \textit{start} : \texttt{Config} \hfill (initial joint configuration)
  \item \textit{goal} : \texttt{Config} \hfill (target joint configuration)
  \item \textit{rParam} : \texttt{RobotParams} \hfill (link lengths, joint limits)
  \item \textit{obs} : \texttt{Obstacles} \hfill (environment obstacles)
\end{itemize}

\subsubsection{Assumptions}
\begin{itemize}
  \item All four parameters are either:
    \begin{enumerate}
      \item successfully parsed from a user‑supplied configuration file, or
      \item provided interactively through a GUI/CLI before the first call to
            \texttt{getAll}.
    \end{enumerate}
  \item Basic validity checks (dimension consistency, non‑negative link
        lengths, joint limits with $\ell < u$, etc.) have already succeeded.
\end{itemize}

\subsubsection{Access‑Routine Semantics}

\paragraph{\texttt{getAll()}}
\begin{itemize}
  \item \textbf{output:} the tuple
        $(\textit{start},\textit{goal},\textit{rParam},\textit{obs})$.
  \item \textbf{exception:} \texttt{ParamErr}  
        \begin{itemize}
          \item if any of the four internal variables is \emph{undefined}, or
          \item if a post‑validation step detects that the parameters are
                mutually inconsistent (e.g.\ mismatch between the length of
                \textit{start} and the number of links in \textit{rParam}).
        \end{itemize}
\end{itemize}

%---------------------------------------------------------------------
\subsubsection{Local Functions}
\begin{description}
  \item[\texttt{parseFile}$(f)$] reads JSON/CSV input file $f$ and initialises
        \textit{start}, \textit{goal}, \textit{rParam}, \textit{obs}.
  \item[\texttt{validate}$(s, g, r, o)$] returns \texttt{true} iff the four
        arguments satisfy dimension and range constraints.
\end{description}

\newpage



\bibliographystyle {plainnat}
\bibliography {../../../refs/References}


% \section{Appendix} \label{Appendix}

% \wss{Extra information if required}

% \newpage{}

% \section*{Appendix --- Reflection}

% \wss{Not required for CAS 741 projects}

% The information in this section will be used to evaluate the team members on the
% graduate attribute of Problem Analysis and Design.

% \input{../../Reflection.tex}

% \begin{enumerate}
%   \item What went well while writing this deliverable? 
%   \item What pain points did you experience during this deliverable, and how
%     did you resolve them?
%   \item Which of your design decisions stemmed from speaking to your client(s)
%   or a proxy (e.g. your peers, stakeholders, potential users)? For those that
%   were not, why, and where did they come from?
%   \item While creating the design doc, what parts of your other documents (e.g.
%   requirements, hazard analysis, etc), it any, needed to be changed, and why?
%   \item What are the limitations of your solution?  Put another way, given
%   unlimited resources, what could you do to make the project better? (LO\_ProbSolutions)
%   \item Give a brief overview of other design solutions you considered.  What
%   are the benefits and tradeoffs of those other designs compared with the chosen
%   design?  From all the potential options, why did you select the documented design?
%   (LO\_Explores)
% \end{enumerate}


\end{document}