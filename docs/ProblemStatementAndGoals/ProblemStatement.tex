\documentclass{article}

\usepackage{tabularx}
\usepackage{booktabs}

\title{Problem Statement and Goals\\2D Robot Arms for Path Planning}

\author{Ziyang Fang}

\date{}

\begin{document}

\maketitle

\begin{table}[hp]
\caption{Revision History} \label{TblRevisionHistory}
\begin{tabularx}{\textwidth}{llX}
\toprule
\textbf{Date} & \textbf{Developer(s)} & \textbf{Change}\\
\midrule
2025-01-20 & Ziyang Fang & Initial the document.\\
\bottomrule
\end{tabularx}
\end{table}

\section{Problem Statement}

\subsection{Problem}
The project involves developing a framework for distance-geometric representation of 2D (planar) robot arms to facilitate path planning in environments with circular obstacles. Unlike traditional inverse kinematics (IK) methods, the proposed approach uses a geometric graph representation of the manipulator and formulates the IK problem as a distance-based graph problem. The goal is to determine the joint positions and angles along a feasible path that avoids obstacles and satisfies the goal configuration.

\subsection{Inputs and Outputs}
\textbf{Inputs:}
\begin{itemize}
    \item Environment: Circular obstacles defined by positions (x,y) and radius.
    \item Robot arm: Number of joints and their respective lengths.
    \item Initial pose: Configuration of the robot arm at the start.
    \item Goal: Desired end-effector position or pose.
    \item Number of poses: Initial value for the number of intermediate poses along the path.
\end{itemize}

\textbf{Outputs:}
\begin{itemize}
    \item Distance-geometric representation: Joint positions at each time step.
    \item Angular representation: Joint angles along the path to the goal.
\end{itemize}

\subsection{Stakeholders}
\begin{itemize}
    \item Researchers and developers in robotics and motion planning.
    \item Robotics engineers working on manipulators and autonomous systems.
    \item End users interested in robust and scalable IK solutions.
\end{itemize}

\subsection{Environment}
\textbf{Hardware:} Simulation environment.\\
\textbf{Software:} Python or MATLAB-based implementation, integrating with existing libraries include GraphIK and Convex Iteration-based solvers.

\section{Goals}
\begin{itemize}
    \item Develop a framework that generalizes to various 2D robotic manipulators.
    \item Ensure accurate and efficient path planning through distance-geometric representations.
    \item Validate the approach using benchmark problems and compare with traditional IK solvers.
\end{itemize}

\section{Stretch Goals}
\begin{itemize}
    \item Extend the framework to include angular velocity and acceleration calculations.
    \item Incorporate additional obstacle types or dynamic obstacles.
    \item Optimize the algorithm for real-time applications.
    \item Extend the framework to handle 3D robot arms, enabling path planning in three-dimensional space.
    \item Enable real-world testing on physical robotic systems to validate the algorithm’s practical applicability and performance.
    \item Achieve higher computational efficiency compared to existing solvers, demonstrating faster convergence and lower computational resource usage.
\end{itemize}


\section{Challenge Level and Extras}
\textbf{Challenge Level:} General.\\
\textbf{Rationale:}
\begin{itemize}
    \item This is a problem with a general-level challenge. 
    \item It requires a solid understanding of robotics kinematics, inverse kinematics solving techniques, and convex optimization theories. 
    \item Solving this problem involves integrating these theoretical foundations into a path-planning framework for planar manipulators, with a focus on handling complex configurations and constrained environments effectively.
\end{itemize}

\textbf{Extras:}
\begin{itemize}
    \item Provide user manual for researchers and engineers.
\end{itemize}

\end{document}
